% Options for packages loaded elsewhere
\PassOptionsToPackage{unicode}{hyperref}
\PassOptionsToPackage{hyphens}{url}
%
\documentclass[
]{article}
\usepackage{lmodern}
\usepackage{amssymb,amsmath}
\usepackage{ifxetex,ifluatex}
\ifnum 0\ifxetex 1\fi\ifluatex 1\fi=0 % if pdftex
  \usepackage[T1]{fontenc}
  \usepackage[utf8]{inputenc}
  \usepackage{textcomp} % provide euro and other symbols
\else % if luatex or xetex
  \usepackage{unicode-math}
  \defaultfontfeatures{Scale=MatchLowercase}
  \defaultfontfeatures[\rmfamily]{Ligatures=TeX,Scale=1}
\fi
% Use upquote if available, for straight quotes in verbatim environments
\IfFileExists{upquote.sty}{\usepackage{upquote}}{}
\IfFileExists{microtype.sty}{% use microtype if available
  \usepackage[]{microtype}
  \UseMicrotypeSet[protrusion]{basicmath} % disable protrusion for tt fonts
}{}
\makeatletter
\@ifundefined{KOMAClassName}{% if non-KOMA class
  \IfFileExists{parskip.sty}{%
    \usepackage{parskip}
  }{% else
    \setlength{\parindent}{0pt}
    \setlength{\parskip}{6pt plus 2pt minus 1pt}}
}{% if KOMA class
  \KOMAoptions{parskip=half}}
\makeatother
\usepackage{xcolor}
\IfFileExists{xurl.sty}{\usepackage{xurl}}{} % add URL line breaks if available
\IfFileExists{bookmark.sty}{\usepackage{bookmark}}{\usepackage{hyperref}}
\hypersetup{
  hidelinks,
  pdfcreator={LaTeX via pandoc}}
\urlstyle{same} % disable monospaced font for URLs
\usepackage[margin=1in]{geometry}
\usepackage{longtable,booktabs}
% Correct order of tables after \paragraph or \subparagraph
\usepackage{etoolbox}
\makeatletter
\patchcmd\longtable{\par}{\if@noskipsec\mbox{}\fi\par}{}{}
\makeatother
% Allow footnotes in longtable head/foot
\IfFileExists{footnotehyper.sty}{\usepackage{footnotehyper}}{\usepackage{footnote}}
\makesavenoteenv{longtable}
\usepackage{graphicx,grffile}
\makeatletter
\def\maxwidth{\ifdim\Gin@nat@width>\linewidth\linewidth\else\Gin@nat@width\fi}
\def\maxheight{\ifdim\Gin@nat@height>\textheight\textheight\else\Gin@nat@height\fi}
\makeatother
% Scale images if necessary, so that they will not overflow the page
% margins by default, and it is still possible to overwrite the defaults
% using explicit options in \includegraphics[width, height, ...]{}
\setkeys{Gin}{width=\maxwidth,height=\maxheight,keepaspectratio}
% Set default figure placement to htbp
\makeatletter
\def\fps@figure{htbp}
\makeatother
\setlength{\emergencystretch}{3em} % prevent overfull lines
\providecommand{\tightlist}{%
  \setlength{\itemsep}{0pt}\setlength{\parskip}{0pt}}
\setcounter{secnumdepth}{-\maxdimen} % remove section numbering

\author{}
\date{\vspace{-2.5em}}

\begin{document}

\textbf{Please read the Rmd file accompanying this PDF, and as you do
so, you can refer to the PDF to see how each example is rendered (or
not).}

\hypertarget{intro}{%
\section{Intro}\label{intro}}

R markdown files allow you to show code and outputs in the order they
were run. However, our professor doesn't want to see our R code until
the end of the report, in an appendix. So, she has said that our reports
\emph{should not} be compiled from R markdown files. But, there is a way
to create PDF reports from R markdown files where the code echoing is
suppressed and instead shown in an appendix! This Rmd file is an example
of that.

I'll show a bunch of example code chunks so you can see the different
options.

\hypertarget{setup-chunk}{%
\subsection{Setup Chunk}\label{setup-chunk}}

Please notice above the \texttt{setup} chunk. There are a couple of
things I want to point out:

\begin{itemize}
\tightlist
\item
  The chunk options are very different from what you are used to
\item
  Every package required anywhere in the report is loaded right up front
\item
  The \texttt{setup} chunk \emph{is not} included in the appendix! It is
  reserved solely for code that is required to facilitate document
  generation
\end{itemize}

So, why have I put \texttt{library} statements there? You'll see that
the \texttt{library} statements are wrapped in
\texttt{suppressPackageStartupMessages} and that I've passed a few extra
parameters that you may not have seen before. This means that packages
will not produce any pesky output in your report when they are loaded.
However, because we don't want to include the \texttt{setup} chunk in
the appendix, you will want to ``re-load'' every package within code
chunks that \emph{will} end up in the appendix.

\hypertarget{a-note-about-default-chunk-options}{%
\subsection{A note about default chunk
options}\label{a-note-about-default-chunk-options}}

You can ignore this section on the first read. Just follow the
conventions outlined below for the different examples.

\begin{longtable}[]{@{}ll@{}}
\toprule
\begin{minipage}[b]{0.24\columnwidth}\raggedright
Default option\strut
\end{minipage} & \begin{minipage}[b]{0.70\columnwidth}\raggedright
Why?\strut
\end{minipage}\tabularnewline
\midrule
\endhead
\begin{minipage}[t]{0.24\columnwidth}\raggedright
\texttt{eval\ =\ TRUE}\strut
\end{minipage} & \begin{minipage}[t]{0.70\columnwidth}\raggedright
All R code is executed by default\strut
\end{minipage}\tabularnewline
\begin{minipage}[t]{0.24\columnwidth}\raggedright
\texttt{echo\ =\ FALSE}\strut
\end{minipage} & \begin{minipage}[t]{0.70\columnwidth}\raggedright
Do not show R code at the time it is run\strut
\end{minipage}\tabularnewline
\begin{minipage}[t]{0.24\columnwidth}\raggedright
\texttt{message\ =\ FALSE}\strut
\end{minipage} & \begin{minipage}[t]{0.70\columnwidth}\raggedright
Do not show any messages\strut
\end{minipage}\tabularnewline
\begin{minipage}[t]{0.24\columnwidth}\raggedright
\texttt{error\ =\ FALSE}\strut
\end{minipage} & \begin{minipage}[t]{0.70\columnwidth}\raggedright
Do not show any warnings\strut
\end{minipage}\tabularnewline
\begin{minipage}[t]{0.24\columnwidth}\raggedright
\texttt{warning\ =\ FALSE}\strut
\end{minipage} & \begin{minipage}[t]{0.70\columnwidth}\raggedright
Do not show any errors\strut
\end{minipage}\tabularnewline
\begin{minipage}[t]{0.24\columnwidth}\raggedright
\texttt{purl\ =\ FALSE}\strut
\end{minipage} & \begin{minipage}[t]{0.70\columnwidth}\raggedright
By default, code chunks \emph{will not} appear in the appendix. You will
have to explicitly mark the ones you want to include\strut
\end{minipage}\tabularnewline
\begin{minipage}[t]{0.24\columnwidth}\raggedright
\texttt{results\ =\ \textquotesingle{}hide\textquotesingle{}}\strut
\end{minipage} & \begin{minipage}[t]{0.70\columnwidth}\raggedright
You are probably used to code chunks outputing something to include in
your report. If you want this, you'll have to explicitly override this
option!\strut
\end{minipage}\tabularnewline
\bottomrule
\end{longtable}

\hypertarget{examples-of-different-configurations}{%
\section{Examples of different
configurations}\label{examples-of-different-configurations}}

\hypertarget{example-1-data-prep-chunk}{%
\subsection{Example 1: data prep
chunk}\label{example-1-data-prep-chunk}}

You'll use this kind of code chunk when you are prepping data for use in
other chunks, but there won't be any output to the report. You want the
code in the appendix so the reader can reproduce your work, but there
isn't any output yet.

Chunk options:

\begin{itemize}
\tightlist
\item
  Default options apply
\item
  \texttt{purl=TRUE} means ``include in appendix''
\end{itemize}

\hypertarget{example-2-content-chunk}{%
\subsection{Example 2: content chunk}\label{example-2-content-chunk}}

The option \texttt{results=\textquotesingle{}markup\textquotesingle{}}
is what you are used to working with in Rmd files. There are other
values you can set \texttt{results} to, but you probably won't use them
very often. (Except for \texttt{asis}, and you will see an example of
that below when we bootstrap in the appendix.)

\includegraphics{report-code-appendix_files/figure-latex/unnamed-chunk-3-1.pdf}

\hypertarget{example-3-kable-output}{%
\subsection{\texorpdfstring{Example 3: \texttt{kable}
output}{Example 3: kable output}}\label{example-3-kable-output}}

Let's say you want to put some table output in your report. But, you
want the reader, when they run your code, to be able to get readable
output. (Nicely formatted stuff will have a lot of extra tags around it
and isn't always the easiest to read.)

\hypertarget{example-4-experiments}{%
\subsection{Example 4: experiments}\label{example-4-experiments}}

You're going to try lots of stuff when you are writing your report. But,
why should you have to delete the code just because it ended up not
being needed?

Remember \texttt{purl=FALSE} and
\texttt{results=\textquotesingle{}hide\textquotesingle{}} are set by
default.

\hypertarget{example-5-code-for-the-reader}{%
\subsection{Example 5: code for the
reader}\label{example-5-code-for-the-reader}}

The following chunk won't do anything for your report or analysis, but
will show up in the appendix. This might be used for something that you
experimented with and talked about, but doesn't have any content for
your report. The reader might want to see what you tried if you've
mentioned it in your write-up.

\newpage

\hypertarget{appendix-1-r-code-for-analysis}{%
\section{Appendix 1: R Code for
Analysis}\label{appendix-1-r-code-for-analysis}}

And, here is the appendix. I haven't figured out how to get the file
name of the Rmd file knitr is compiling, so that is hardcoded. (It's the
name of this Rmd file!)

\begin{verbatim}
# ============================
# Example 1: data prep chunk
# ============================

# Re-list the packages your code uses
# You don't need to list knitr unless that is required for reproducing your work
###library(alrtools)
library(tidyverse)

# Notice that I've put a big banner comment at the beginning of this
# Since I am including it in the appendix, I want the reader to be
# able to know what section of the report the code applies to

# If you are using functions the reader may not have seen before
# it's not a bad idea to preface them with the package they come from.
# readr was loaded as part of the tidyverse
# So the "namespacing" is not required, only helpful
###boston <- readr::read_csv('crime-training-data_modified.csv')

# ============================
# Example 2: data prep chunk
# ============================

###mod1 <- lm(medv ~ age + rm, data = boston)
###par(mfrow = c(2, 2))
plot(cars)

# ============================
# Example 3: `kable` output
# ============================

# This shows a table of response variable versus rounded room counts
# But, it's not pretty
###tbl <- table(boston$target, round(boston$rm, 0))
print(cars)





# ============================
# Example 5: code for the reader
# ============================
###library(tree)
###tree1 <- tree::tree(medv ~ ., data = boston)
###par(mfrow = c(1, 1))
###plot(tree1, type = 'uniform')
###text(tree1, pretty = 5, col = 'blue', cex = 0.8)
\end{verbatim}

\end{document}
